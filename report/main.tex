%!TeX program = xelatex
\documentclass[12pt,hyperref,a4paper,UTF8]{ctexart}
\usepackage{zjureport}

%%-------------------------------正文开始---------------------------%%
\begin{document}

%%-----------------------封面--------------------%%
\cover

%%------------------摘要-------------%%
%\begin{abstract}
%
%在此填写摘要内容
%
%\end{abstract}

\thispagestyle{empty} % 首页不显示页码

%%--------------------------目录页------------------------%%
% \newpage
% \tableofcontents

%%------------------------正文页从这里开始-------------------%
\newpage

%%可选择这里也放一个标题
%\begin{center}
%    \title{ \Huge \textbf{{标题}}}
%\end{center}

\section{实验目的和要求}
\begin{itemize}
    \item 掌握周转时间、等待时间、平均周转时间等概念及其计算方法。
    \item 理解三种常用的进程调度算法,区分算法之间的差异性,并模拟实现各算法。
    \item 了解操作系统中高级调度、中级调度和低级调度的区别和联系
\end{itemize}

\section{问题描述}
    \begin{itemize}
        \item (1)在单道环境下,已知 $n$个作业的进入时间和估计运行时间(以分钟计),分别求出每一个作业的开始时间、结束时间、周转时间、带权周转时间,以及这些作业的平均周转时间和带权平均周转时间;
        \item (2)在多道环境(如 2 道)下,已知 $n$ 个作业的进入时间和估计运行时间(以分钟计),分别求出每一个作业的开始时间、结束时间、周转时间、带权周转时间,以及这些作业的平均周转时间和带权平均周转时间。
    \end{itemize}

\section{实验要求}
    \begin{itemize}
        \item 分别用先来先服务调度算法(FCFS)、短作业优先调度算法(SJF)、响应比高者优先调度算法(HRRN),求出批作业的平均周转时间和带权平均周转时间;
        \item 就同一批次作业,分别讨论这些算法的优劣;
        \item 衡量同一调度算法对不同作业流的性能。
    \end{itemize}


\section{实验环境}
    \begin{itemize}
        \item 开发工具:VS Code
        \item 编程语言: Rust
    \end{itemize}

\section{设计思想及实验步骤}
(包括实验设计原理,分析方法、计算步骤、模块组织,或主要流程图、伪代码等)

\subsection{实验设计原理}
本实验采用非抢占式批处理作业调度模型,核心指标:周转时间 \(T_i=C_i-A_i\)、带权周转时间 \(W_i=T_i/S_i\),以及其平均值 \(\overline{T}=\frac{1}{n}\sum_i T_i\)、\(\overline{W}=\frac{1}{n}\sum_i W_i\)。多道情形抽象为 \(m\) 条并行“道”,作业一旦开始在某道上运行至完成。

\subsection{分析方法}
实现三种典型非抢占调度算法:FCFS(按到达时间先后分配到最早空闲道)、SJF(就绪集合选服务时间最短者)、HRRN(就绪集合按 \(R=(t-A+S)/S\) 从大到小选择)。时间推进遵循事件驱动:若有空闲道且就绪非空则立即分配,否则跳至下一个到达或最近空闲时间的较小者。

\subsection{计算步骤}
\begin{enumerate}
    \item 构造作业流 \((A_i,S_i)\),设定道数 \(m\);
    \item 维护就绪集合与各道最早空闲时间;
    \item 依据所选算法分配作业并记录开始/结束时间;
    \item 计算 \(T_i, W_i\) 及 \(\overline{T}, \overline{W}\);
    \item 输出并整理为表格,进行对比分析。
\end{enumerate}

\subsection{模块组织}
Rust 程序包含:结构体 `Job`;函数 `schedule\_fcfs`、`schedule\_sjf`、`schedule\_hrrn`;`print\_results` 统计与打印;`main` 组装单道/双道与两组作业流实验。

\subsection{伪代码}
以 SJF 为例(非抢占,多道):
\begin{verbatim}
time <- 0; core_free[m] <- 0; ready <- {}
按到达时间排序 all
while 未完成:
  将 arrival <= time 的作业加入 ready
  free_cores <- {k | core_free[k] <= time}
  if free_cores 为空:
    if ready 非空: time <- 最早 core_free
    else: time <- min(下一个到达, 最早 core_free); continue
  if ready 为空: time <- 下一个到达; continue
  按 service 升序排序 ready
  对每个空闲 core:
    取最短作业 j;start <- time; end <- start + service
    记录 j.start/j.end;core_free[core] <- end;加入 finished
  time <- min(下一个到达, 最早 core_free)
\end{verbatim}

\section{实验结果及分析}
以下给出程序运行结果(单位:分钟)。

\paragraph{原始终端输出}
\begin{verbatim}
=== FCFS - 单道 ===
id      arr     serv    start   end     turn    wturn
1       0.00    3.00    0.00    3.00    3.00    1.00
2       2.00    6.00    3.00    9.00    7.00    1.17
3       4.00    4.00    9.00    13.00   9.00    2.25
4       6.00    5.00    13.00   18.00   12.00   2.40
5       8.00    2.00    18.00   20.00   12.00   6.00
平均周转时间 = 8.6000
带权平均周转时间 = 2.5633

=== SJF - 单道 ===
id      arr     serv    start   end     turn    wturn
1       0.00    3.00    0.00    3.00    3.00    1.00
2       2.00    6.00    3.00    9.00    7.00    1.17
3       4.00    4.00    11.00   15.00   11.00   2.75
4       6.00    5.00    15.00   20.00   14.00   2.80
5       8.00    2.00    9.00    11.00   3.00    1.50
平均周转时间 = 7.6000
带权平均周转时间 = 1.8433

=== HRRN - 单道 ===
id      arr     serv    start   end     turn    wturn
1       0.00    3.00    0.00    3.00    3.00    1.00
2       2.00    6.00    3.00    9.00    7.00    1.17
3       4.00    4.00    9.00    13.00   9.00    2.25
4       6.00    5.00    15.00   20.00   14.00   2.80
5       8.00    2.00    13.00   15.00   7.00    3.50
平均周转时间 = 8.0000
带权平均周转时间 = 2.1433

=== FCFS - 双道 ===
id      arr     serv    start   end     turn    wturn
1       0.00    3.00    0.00    3.00    3.00    1.00
2       2.00    6.00    2.00    8.00    6.00    1.00
3       4.00    4.00    4.00    8.00    4.00    1.00
4       6.00    5.00    8.00    13.00   7.00    1.40
5       8.00    2.00    8.00    10.00   2.00    1.00
平均周转时间 = 4.4000
带权平均周转时间 = 1.0800

=== SJF - 双道 ===
id      arr     serv    start   end     turn    wturn
1       0.00    3.00    0.00    3.00    3.00    1.00
2       2.00    6.00    2.00    8.00    6.00    1.00
3       4.00    4.00    4.00    8.00    4.00    1.00
4       6.00    5.00    8.00    13.00   7.00    1.40
5       8.00    2.00    8.00    10.00   2.00    1.00
平均周转时间 = 4.4000
带权平均周转时间 = 1.0800

=== HRRN - 双道 ===
id      arr     serv    start   end     turn    wturn
1       0.00    3.00    0.00    3.00    3.00    1.00
2       2.00    6.00    2.00    8.00    6.00    1.00
3       4.00    4.00    4.00    8.00    4.00    1.00
4       6.00    5.00    8.00    13.00   7.00    1.40
5       8.00    2.00    8.00    10.00   2.00    1.00
平均周转时间 = 4.4000
带权平均周转时间 = 1.0800

=== 同一算法在不同作业流上的比较(示例) ===

=== Stream A - FCFS - 单道 ===
id      arr     serv    start   end     turn    wturn
1       0.00    3.00    0.00    3.00    3.00    1.00
2       2.00    6.00    3.00    9.00    7.00    1.17
3       4.00    4.00    9.00    13.00   9.00    2.25
4       6.00    5.00    13.00   18.00   12.00   2.40
5       8.00    2.00    18.00   20.00   12.00   6.00
平均周转时间 = 8.6000
带权平均周转时间 = 2.5633

=== Stream B - FCFS - 单道 ===
id      arr     serv    start   end     turn    wturn
1       0.00    8.00    0.00    8.00    8.00    1.00
2       1.00    4.00    8.00    12.00   11.00   2.75
3       2.00    9.00    12.00   21.00   19.00   2.11
4       3.00    5.00    21.00   26.00   23.00   4.60
5       10.00   2.00    26.00   28.00   18.00   9.00
6       10.00   1.00    28.00   29.00   19.00   19.00
平均周转时间 = 16.3333
带权平均周转时间 = 6.4102
(base) ➜  lab1 git:(main) ✗ 
\end{verbatim}

\subsection{单道(m=1)}
\paragraph{FCFS}
\begin{table}[!htbp]
\centering
\begin{tabular}{c|c|c|c|c|c|c}
id & arr & serv & start & end & turn & wturn \\
\hline
1 & 0.00 & 3.00 & 0.00 & 3.00 & 3.00 & 1.00 \\
2 & 2.00 & 6.00 & 3.00 & 9.00 & 7.00 & 1.17 \\
3 & 4.00 & 4.00 & 9.00 & 13.00 & 9.00 & 2.25 \\
4 & 6.00 & 5.00 & 13.00 & 18.00 & 12.00 & 2.40 \\
5 & 8.00 & 2.00 & 18.00 & 20.00 & 12.00 & 6.00 \\
\hline
\multicolumn{7}{r}{\small 平均周转时间 = 8.6000,带权平均周转时间 = 2.5633}
\end{tabular}
\end{table}

\paragraph{SJF}
\begin{table}[!htbp]
\centering
\begin{tabular}{c|c|c|c|c|c|c}
id & arr & serv & start & end & turn & wturn \\
\hline
1 & 0.00 & 3.00 & 0.00 & 3.00 & 3.00 & 1.00 \\
2 & 2.00 & 6.00 & 3.00 & 9.00 & 7.00 & 1.17 \\
3 & 4.00 & 4.00 & 11.00 & 15.00 & 11.00 & 2.75 \\
4 & 6.00 & 5.00 & 15.00 & 20.00 & 14.00 & 2.80 \\
5 & 8.00 & 2.00 & 9.00 & 11.00 & 3.00 & 1.50 \\
\hline
\multicolumn{7}{r}{\small 平均周转时间 = 7.6000,带权平均周转时间 = 1.8433}
\end{tabular}
\end{table}

\paragraph{HRRN}
\begin{table}[!htbp]
\centering
\begin{tabular}{c|c|c|c|c|c|c}
id & arr & serv & start & end & turn & wturn \\
\hline
1 & 0.00 & 3.00 & 0.00 & 3.00 & 3.00 & 1.00 \\
2 & 2.00 & 6.00 & 3.00 & 9.00 & 7.00 & 1.17 \\
3 & 4.00 & 4.00 & 9.00 & 13.00 & 9.00 & 2.25 \\
4 & 6.00 & 5.00 & 15.00 & 20.00 & 14.00 & 2.80 \\
5 & 8.00 & 2.00 & 13.00 & 15.00 & 7.00 & 3.50 \\
\hline
\multicolumn{7}{r}{\small 平均周转时间 = 8.0000,带权平均周转时间 = 2.1433}
\end{tabular}
\end{table}

\subsection{双道(m=2)}
三种算法在该作业流下得到相同的调度与指标:
\begin{table}[!htbp]
\centering
\begin{tabular}{c|c|c|c|c|c|c}
id & arr & serv & start & end & turn & wturn \\
\hline
1 & 0.00 & 3.00 & 0.00 & 3.00 & 3.00 & 1.00 \\
2 & 2.00 & 6.00 & 2.00 & 8.00 & 6.00 & 1.00 \\
3 & 4.00 & 4.00 & 4.00 & 8.00 & 4.00 & 1.00 \\
4 & 6.00 & 5.00 & 8.00 & 13.00 & 7.00 & 1.40 \\
5 & 8.00 & 2.00 & 8.00 & 10.00 & 2.00 & 1.00 \\
\hline
\multicolumn{7}{r}{\small 平均周转时间 = 4.4000,带权平均周转时间 = 1.0800}
\end{tabular}
\end{table}

\subsection{讨论与分析}
\begin{itemize}
    \item 单道下:SJF 的 \(\overline{T},\overline{W}\) 最低;HRRN 介于 FCFS 与 SJF 之间,能够缓解长作业饥饿;
    \item 双道下:由于到达/服务时间结构,本例三算法输出一致,且显著优于单道;
    \item 多道并行能降低等待时间;真实系统若考虑 I/O、抢占、优先级等,策略需进一步扩展。
\end{itemize}

\section{附录:部分源代码}
\begin{verbatim}
use std::cmp::Ordering;

#[derive(Clone, Debug)]
struct Job {
    id: usize,
    arrival: f64, // 到达时间,分钟
    service: f64, // 估计运行时间,分钟
    start: Option<f64>,
    end: Option<f64>,
}

impl Job {
    fn new(id: usize, arrival: f64, service: f64) -> Self {
        Self { id, arrival, service, start: None, end: None }
    }

    fn turnaround(&self) -> Option<f64> {
        match (self.end, Some(self.arrival)) {
            (Some(e), Some(a)) => Some(e - a),
            _ => None,
        }
    }

    fn weighted_turnaround(&self) -> Option<f64> {
        match (self.turnaround(), self.service) {
            (Some(t), s) if s > 0.0 => Some(t / s),
            _ => None,
        }
    }
}

\end{verbatim}


\section{写在最后}
\subsection{发布地址}
\begin{itemize}
    \item Github: \url{https://github.com/eleliauk/OS-LABS}
\end{itemize}

%%----------- 参考文献 -------------------%%
%在reference.bib文件中填写参考文献,此处自动生成

% \reference


\end{document}